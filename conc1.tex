%!TEX root = CSL.tex

This research line offers at least four promising directions for future exploration.

First, the work initiated in~\cite{DBLP:conf/tableaux/LangOPF19} highlights that our games and systems provide more precise control over resources appearing negatively in sequents, unlocking new opportunities for analyzing the problem of comparing proofs. For instance, studying proof costs in labeled calculi could reveal deeper links between labels and computational bounds~\cite{DBLP:journals/jfp/AccattoliGK20}. Similarly, examining the interplay between resource budgets and the complexity of the cut-elimination process, particularly within the multiplicative-(sub)exponential fragment, holds significant potential~\cite{DBLP:journals/tcs/Strassburger03,DBLP:journals/tocl/StrassburgerG11}.

Second, there is substantial value in investigating how the dialogue games we have developed align with the framework of concurrent games~\cite{DBLP:conf/lics/AbramskyM99,DBLP:conf/lics/FaggianM05,DBLP:journals/lmcs/CastellanCRW17}. Understanding these connections could enrich our framework and provide new perspectives on resource management in proof theory.

\blue{EP. Semirings?}

Lastly, an essential direction involves addressing compositionality in dialogue games governed by the cut rule. Regardless of the specific approach taken to achieve cut-admissibility, ensuring compositionality remains a critical and promising challenge~\cite{dutilh18}.