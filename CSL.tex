\documentclass[a4paper,UKenglish,cleveref, autoref, thm-restate]{lipics-v2021}

\bibliographystyle{plainurl}% the mandatory bibstyle

% shortcuts
\newcommand{\I}{\emph{I}\xspace}
\newcommand{\You}{\emph{You}\xspace}
\newcommand{\My}{\emph{My}\xspace}
\newcommand{\Myself}{\emph{Myself}\xspace}
\newcommand{\Me}{\emph{Me}\xspace}
\newcommand{\Your}{\emph{Your}\xspace}

%\newcommand{\DS}{\mathbf{DS}^{HPNL}}

\newcommand{\DS}{\mathbf{DS}}
\newcommand{\dDS}{\mathbf{dDS}}
\newcommand{\DG}{\mathbf{DG}}

\newcommand{\At}{\mathsf{At}}
\renewcommand{\emptyset}{\varnothing}

\newcommand{\red}[1]{\textcolor{red}{#1}}
\newcommand{\blue}[1]{\textcolor{blue}{#1}}

\newcommand\eg{\hbox{\textit{e.g.}}}
\newcommand\ie{\hbox{\textit{i.e.}}}
\newcommand{\etal}{\emph{et al.}}
\newcommand{\cf}{{\em cf.}}

\newcommand{\seq}{\Rightarrow}
\newcommand{\rs}[3]{#1;#2\seq #3}
\newcommand{\rc}{\mathcal{R}}

%Modal Logic abbreviations
\newcommand{\M}{\mathbb{M}}
\newcommand{\N}{\mathbb{N}}
\newcommand{\IM}{\mathcal{I}}
\newcommand{\A}{\mathsf{A}}
\newcommand{\R}{\mathsf{R}}
\newcommand{\V}{\mathsf{V}}
\newcommand{\g}{\mathsf{g}}
\newcommand{\ag}{\mathsf{a}}
\renewcommand{\b}{\mathsf{b}}
\renewcommand{\c}{\mathsf{c}}
\newcommand{\f}{\mathsf{f}}

\renewcommand{\qed}{\hfill$\blacksquare$}

%\newtheorem{example}{Example}
%\newtheorem{lemma}{Lemma}
%\newtheorem{notation}{Notation}
%\newtheorem{theorem}{Theorem}
%\newtheorem{definition}{Definition}
%\newtheorem{proposition}{Proposition}
%\newtheorem{remark}{Remark}
%\newtheorem{corollary}{Corollary}
%\renewenvironment{proof}{\noindent\textit{Proof:}\quad}{\qed}

% Some PNL operators
\newcommand{\pnlP}{\langle \bigdoublewedge+ \rangle }
\newcommand{\pnlN}{\langle\bigdoublewedge-\rangle}
\newcommand{\pnlPN}{\langle\bigdoublewedge\pm\rangle}
\newcommand{\pnlOP}{\langle\oplus\rangle}
\newcommand{\pnlON}{\langle\ominus\rangle}
\newcommand{\dplus}{\meddiamondplus}
%\newcommand{\dplus}{\ensurestackMath{%
 % \stackengine{.8pt}{\Diamond}{\scalebox{.9}[1]{$+$}}{O}{c}{F}{F}{L}}}
\newcommand{\dminus}{\meddiamondminus}
%\newcommand{\dminus}{\ensurestackMath{%
  %\stackengine{.8pt}{\Diamond}{\scalebox{.9}[1]{$-$}}{O}{c}{F}{F}{L}}}
\newcommand{\bplus}{\boxplus}
\newcommand{\bminus}{\boxminus}
\newcommand{\dplusminus}{\Diamond^{\pm}}


\newcommand{\pP}{\wedge\hspace{-0.25cm}\wedge\!\!+ }
\newcommand{\pN}{\wedge\hspace{-0.25cm}\wedge\!\!-}
\newcommand{\pPN}{\wedge\hspace{-0.25cm}\wedge\!\!\pm}

\DeclareMathOperator*{\bigdoublewedge}{\wedge\mkern-15mu\wedge}

% Some shorthands 
\newcommand{\bbM}{\mathbb{M}}
\newcommand{\bfP}{\mathbf{P}}
\newcommand{\bfO}{\mathbf{O}}
\newcommand{\Rinv}{R_{\leftrightarrow}}

% Classes of models
\newcommand{\ccpnlmodels}{\mathfrak{M}_{\textit{ccPNL}}}
\newcommand{\pnlmodels}{\mathfrak{M}_{\textit{PNL}}}
\newcommand{\namedmodels}{\mathfrak{M}_{\textit{N}}}

\newcommand{\PNL}{\textbf{PNL}}
\newcommand{\ccPNL}{\textbf{ccPNL}}
\newcommand{\cc}{\textbf{cc}}
\newcommand{\dPNL}{\textbf{dPNL}}

% Lines in figures
\def\headline#1{\hbox to \hsize{\hrulefill\quad\lower.3em\hbox{#1}\quad\hrulefill}}

\usepackage{pmboxdraw}
\usepackage{fancyvrb}
\usepackage{xcolor}
\usepackage{etoolbox}
\usepackage{amsmath,amssymb,trimclip,adjustbox}
\usepackage{stackengine,amssymb,graphicx}
\usepackage{color}
\usepackage{colortbl}
\usepackage{xspace}

\usepackage{modalops}
\usepackage{xifthen}

%tables
\usepackage{multicol}
\usepackage{longtable}

\usepackage{ebproof}	%prooftrees

\usepackage{proof}

\usepackage{comment} %comments
\usepackage{tabularx}

\usepackage{cancel}
\usepackage[colorinlistoftodos,prependcaption,textsize=tiny]{todonotes}

\usepackage{stackengine}

\usepackage{cleveref}

\usepackage{doc}
\usepackage{tikz}
\usepackage{todonotes}
\newcommand{\notetwo}[3]{\todo[size=\tiny,color=#1]{\texttt{\color{white} #2: #3}}}
\newcommand\co[1]{\notetwo{orange}{Carlos}{#1}}
\newcommand\ep[1]{\notetwo{blue}{Elaine}{#1}}

\usetikzlibrary{positioning, quotes}

\title{Playing with modalities}

\titlerunning{Playing with modalities}


\author{Elaine Pimentel\footnote{Corresponding author.}}{Computer Science Department UCL, UK \and \url{https://sites.google.com/site/elainepimentel/} }{e.pimentel@ucl.ac.uk}{https://orcid.org/0000-0002-7113-0801}{Pimentel has received funding from the European Union's Horizon 2020 research and innovation programme under the Marie Sk\l odowska-Curie grant agreement Number 101007627 and by the Leverhulme Project ECUMENICAL.}

\author{Carlos Olarte}{LIPN, CNRS UMR 7030, Universit\'{e} Sorbonne Paris Nord, France \and \url{https://sites.google.com/site/carlosolarte/} }{olarte@lipn.univ-paris13.fr}{https://orcid.org/0000-0002-7264-7773}{The work of Olarte has been partially supported by the SGR project PROMUEVA (BPIN
2021000100160) under the supervision of Minciencias (Ministerio de Ciencia Tecnolog\'ia e Innovaci\'on, Colombia). Olarte acknowledges also support from the NATO
Science for Peace
and Security Programme through grant number G6133 (project SymSafe). }

\author{Timo Lang}{Computer Science Department UCL, UK \and \url{https://www.timolang.com/}}{timo.lang@ucl.ac.uk}{0000-0002-8257-968X}{}

\author{Robert Freiman}{TU-Wien, Austria}{robert@logic.at}{0000-0001-8251-4272}{}

\author{Christian G. Ferm\"{u}ller}{TU-Wien, Austria \and \url{https://www.logic.at/staff/chrisf/home.html}}{chrisf@logic.at}{0000-0003-2932-5477}{}

\authorrunning{E. Pimentel, C. Olarte, T. Lang, R. Freiman and C. Fehrm\"{u}ller} %TODO mandatory. First: Use abbreviated first/middle names. Second (only in severe cases): Use first author plus 'et al.'

\Copyright{E. Pimentel, C. Olarte, T. Lang, R. Freiman and C. Fehrm\"{u}ller} %TODO mandatory, please use full first names. LIPIcs license is "CC-BY";  http://creativecommons.org/licenses/by/3.0/

\begin{CCSXML}
<ccs2012>
   <concept>
       <concept_id>10003752.10003790.10003801</concept_id>
       <concept_desc>Theory of computation~Linear logic</concept_desc>
       <concept_significance>500</concept_significance>
       </concept>
   <concept>
       <concept_id>10003752.10003790.10003793</concept_id>
       <concept_desc>Theory of computation~Modal and temporal logics</concept_desc>
       <concept_significance>500</concept_significance>
       </concept>
   <concept>
       <concept_id>10003752.10003790.10003792</concept_id>
       <concept_desc>Theory of computation~Proof theory</concept_desc>
       <concept_significance>500</concept_significance>
       </concept>
 </ccs2012>
\end{CCSXML}

\ccsdesc[500]{Theory of computation~Linear logic}
\ccsdesc[500]{Theory of computation~Modal and temporal logics}
\ccsdesc[500]{Theory of computation~Proof theory}
%
%\ccsdesc[100]{\textcolor{red}{Replace ccsdesc macro with valid one}} %TODO mandatory: Please choose ACM 2012 classifications from https://dl.acm.org/ccs/ccs_flat.cfm 

\keywords{Linear logic, modal logic, proof theory, game semantics} %TODO mandatory; please add comma-separated list of keywords

\category{Invited Talk} 

\relatedversion{} %optional, e.g. full version hosted on arXiv, HAL, or other respository/website
%\relatedversiondetails[linktext={opt. text shown instead of the URL}, cite=DBLP:books/mk/GrayR93]{Classification (e.g. Full Version, Extended Version, Previous Version}{URL to related version} %linktext and cite are optional

%\supplement{}%optional, e.g. related research data, source code, ... hosted on a repository like zenodo, figshare, GitHub, ...
%\supplementdetails[linktext={opt. text shown instead of the URL}, cite=DBLP:books/mk/GrayR93, subcategory={Description, Subcategory}, swhid={Software Heritage Identifier}]{General Classification (e.g. Software, Dataset, Model, ...)}{URL to related version} %linktext, cite, and subcategory are optional

%\funding{(Optional) general funding statement \dots}%optional, to capture a funding statement, which applies to all authors. Please enter author specific funding statements as fifth argument of the \author macro.

%\acknowledgements{I want to thank \dots}

\EventEditors{J\"{o}rg Endrullis and Sylvain Schmitz}
\EventNoEds{2}
\EventLongTitle{33rd EACSL Annual Conference on Computer Science Logic (CSL 2025)}
\EventShortTitle{CSL 2025}
\EventAcronym{CSL}
\EventYear{2025}
\EventDate{February 10--14, 2025}
\EventLocation{Amsterdam, Netherlands}
\EventLogo{}
\SeriesVolume{326}
\ArticleNo{4}

\begin{document}

\maketitle

%TODO mandatory: add short abstract of the document
\begin{abstract}
In this work, we will explore modalities through dialogical game lenses. Games provide a powerful tool for bridging the gap between intended and formal semantics, often offering a more conceptually natural approach to logic than traditional model-theoretic semantics.

We begin by exploring substructural calculi  from a game semantic perspective, driven by intuitions about resource-consciousness and, more specifically, cost-sensitive reasoning. The game comes into full swing as we introduce cost labels to assumptions and a corresponding budget. Different proofs of the same end-sequent are interpreted as strategies for a player to defend a claim, which vary in cost. This leads to a labelled calculus, which can be viewed as a fragment of subexponential linear logic. 
%
We conclude this first part with a discussion of cut-admissibility for the proposed system. 

In the second part, we show that our games offer an interesting insight also into modal logics. More precisely, we will focus on the modal logic \PNL, characterized by Kripke frames with two types of disjoint and symmetric reachability relations. This framework is motivated by the study of group polarization, where the opinions or beliefs of individuals within a group become more extreme or polarized after interaction. Our approach to reasoning about group polarization is based on \PNL\ and highlights a different aspect of formal reasoning about the corresponding models -- using games and proof systems.
%
We conclude by outlining potential directions for future research.

\end{abstract}

\section{Introduction}\label{sec:intro}
%!TEX root = CSL.tex
% !TEX spellcheck = en-UK

%Roadmap:
%\begin{itemize}
%\item Introduce SELL and the labelled game
%\item talk about cut-elimination
%%\item (maybe) generalisations?
%\item Introduce PNL and games
%\item provability games
%\item future work (CK, etc)
%Remember: semirings and extensions, discussion about modal systems, CK etc.
%\end{itemize}

Modalities, both as formal constructs and as tools for reasoning, have been central to the development of logic and proof theory. In this work, we explore modalities through the lens of dialogical games, emphasising their potential to bridge the gap between formal semantics and conceptual intuition. Games not only offer a dynamic perspective on logical systems but also serve as a unifying framework for analysing the structure of proofs and resource management in a variety of logical settings.

We begin by examining substructural calculi, inspired by resource-sensitive reasoning. 
%
%This approach introduces cost labels to assumptions, complemented by a budgetary framework. Different proofs of the same end-sequent are interpreted as strategies with varying costs, enabling the development of a labeled calculus -- a fragment of subexponential linear logic ($\SELL$). 
We introduce the concept of \emph{prices} for resources (represented by formulas) into the game using the unary operator $\nbang{a}$, $a \in \real^+$, which shares some characteristic features with \emph{subexponentials} in \LL\ (\SELL~\cite{DBLP:conf/kgc/DanosJS93,nigam09ppdp}). Intuitively, a formula $\nbang{a}A$ represents a \emph{permanent resource}: from $\nbang{a}A$, we can derive $A$ as many times as needed, paying the price~$a$ each time.

We extend our game to this enriched language by incorporating a \emph{budget} into the game states, which decreases whenever a price is paid. Different strategies for proving the same end-sequent can then be evaluated based on the budget required to execute them safely, \ie, without incurring debt. This approach to resource-consciousness not only enhances the game but also translates naturally into a sequent system, where cost bounds for proofs are expressed as labels attached to sequents. By associating costs with proof steps, we provide a fine-grained analysis of proof strategies and their computational bounds.

We note that, up to this point, the content summarises the work presented in~\cite{DBLP:conf/tableaux/LangOPF19}, where resources were considered only in {\em assumptions}. In this setting, sequents are restricted by limiting the occurrences of the modality $\nbang{a}$ {\em negatively}, thereby eliminating the need for a promotion rule.

In Section~\ref{subsec:cut}, we introduce new perspectives by allowing modalities in positive contexts. This includes the addition of ``worse costs,'' linearisation of the cut formula, and tracking the use of contraction during the cut-elimination process.

In the second part of the paper we go beyond resource-awareness, showing how games can illuminate modal logics. Specifically, we focus on the positive-negative modal logic (\PNL~\cite{DBLP:journals/jolli/XiongA20}), characterised by Kripke frames with two disjoint and symmetric reachability relations. Our interest in \PNL\ stems from its application in modelling phenomena such as group polarisation, where interactions amplify the extremity of opinions within a network. 

In \PNL, individuals in a
social network are identified with worlds of the frame, and they are related
either as ``friends'' (positive) or as ``enemies'' (negative), but not both at
the same time. These relationships can be understood in different ways: Instead
of genuine friendship or enduring enmity, they may simply signify agreement or
disagreement on a particular issue. 

We take inspiration from a work by Pedersen, Smets, and {\AA}gnotes
\cite{DBLP:journals/logcom/PedersenSA21}, where \PNL~is extended in
various ways to axiomatically characterise modally undefinable frame
properties.
 
Here, we present an overview of our work in~\cite{LPAR2024:Reasoning_About_Group_Polarization}. Our approach to logical reasoning about group polarisation is also based on \PNL~but focuses on a different aspect of formal reasoning about the corresponding models through games and proof systems. Specifically, we introduce both a semantic game and a provability game for (hybrid) extensions of \PNL.

In semantic games~\cite{Hintikka1973-HINLLA-2}, each instance is played over a formula $F$ and a model $\M$ by two players, traditionally called \Ic (or \Me) and \You. At every point in the game, one player acts as the proponent ($\mathbf P$), while the other acts as the opponent ($\mathbf O$) of the current formula. The set of actions at each stage is determined by the main connective of the current formula.

In contrast, provability games~\cite{Lorenzen1978-LORDLJ-2} do not concern truth in a specific model but rather {\em logical validity}. These games are also played by two participants, \Me and \You, and involve attacking assertions of formulas made by the other player and defending against these attacks.

We conclude this summary by showing how to transform the semantic game over single models into a provability game that characterises logical validity. This transformation led to {\em the first} Gentzen-style systems for variants of \PNL, which modularly adapt to different frame properties by faithfully capturing the rules for \emph{elementary} games. All the proposed games are accompanied by a prototypical implementation~\cite{tool} using rewriting logic and Maude~\cite{DBLP:journals/jlp/Meseguer12,DBLP:journals/jlap/DuranEEMMRT20}.

Each part concludes with a discussion of future research directions and methodologies for combining and adapting the frameworks presented here to other logics and systems.

\section{A game model for costs}\label{sec:sell}
%!TEX root = CSL.tex

Our starting point is a calculus for \emph{affine intuitionistic linear logic} ($\aILL$)~\cite{DBLP:journals/tcs/Girard87}. Formulas in $\aILL$ are built from the grammar 
\[
 A ::= p \mid \zero \mid \one  
 \mid A_1 \with A_2  \mid A_1 
 \oplus A_2 \mid   A_1 \tensor A_2
  \mid A_1 \lolli A_2\mid \bang A.
\]
with a denumerable infinite set of propositional variables $\{p, q, r, \ldots\}$, the unities\co{units?} $\{\zero,\one\}$,  the binary connectives for additive conjunction and disjunction $\{\with,\oplus\}$, the multiplicative conjunction $\tensor$, the  linear implication $\lolli$ \co{I'd add ``,'' and...}and the exponential $\bang$.

Similar to modal connectives, the exponential $\bang$ in intuitionistic linear logic  is not {\em canonical}~\cite{DBLP:conf/kgc/DanosJS93}, in the sense that, even having the same scheme for introduction rules, marking the exponentials with different labels  does not preserve equivalence. That is, if  $i\not= j$ then
$\nbang{i}A\not\equiv\nbang{j}A$.\co{Link break doesn't seem to be needed here?}

Intuitively, this means that we can mark the exponential with {\em labels} taken from a set $\mathcal{I}$ organized in a pre-order $\preceq$ (\ie, a reflexive and transitive relation), obtaining (possibly infinitely-many) exponentials ($\nbang{i}$
%,\nquest{i}$ 
for $i\in\mathcal{I}$).
Also as in multi-modal systems, the pre-order  determines the provability relation: 
for a general formula $A$, $\nbang{b}A$ {\em implies} $\nbang{a}A$ iff $a \preceq b$.
%
Pre-ordering the labels (together with an upward closeness requirement)
guarantees cut-elimination~\cite{DBLP:journals/jar/NigamM10}. 

The algebraic structure of subexponentials, combined with their intrinsic structural property allow for the proposal of rich linear logic based frameworks. This opened a venue for proposing different multi-modal substructural logical systems, that encountered a number of different applications (see~\cite{DBLP:conf/fscd/PimentelON21} for a survey). 

In this paper, we will use subexponentials to model the notion of {\em costs}. We will start by considering the particular case where labels will be elements of $\real^+$, the set of non-negative real numbers. Formally, we substitute in $\aILL$
%our language 
the exponential $\bang$ by the unary modal operators $\nbang{a}$ 
for each $a\in\real^+$. 
%and call the resulting formulas \emph{extended formulas}. 

%An extended formula $\nbang{a}A$ will be seen as a \emph{permanent resource}: We can obtain as many copies of~$A$ from it as we want, each time paying the price~$a$.

We shall use $A,B,C$ (resp. $\Gamma,\Delta$) to range over formulas (resp. multisets of formulas).  
Sequents have the form $\Gamma\seq C$ where subformulas $\nbang{a}A$ will have a restriction to occur only {\em negatively} in the sequent.\footnote{The notion of polarity  is the standard one: A subformula occurrence in the antecedent of a sequent is {\em negative} if it occurs in the scope of an even number (including $0$) of contexts $([\cdot]\limp B)$, and otherwise it is {\em positive}. For occurrences of a subformula in the consequent, one replaces ``even'' by ``odd''. The reason for this restriction will be made clear in Section~\ref{sec:cut}.}
%
We denote by $\nbang{}\Gamma$ a set of formulas  prefixed with $\nbang{a}$ for some (not necessarily the same) $a\in\real^+$. 

The rules for the system $\aIMALLR$ are depicted in Fig.~\ref{fig:ll}. Note that the cut rule is not included in our presentation of $\aIMALL$ and that weakening is present only implicitly, via the context $\Gamma$ in the initial sequents. Furthermore, in rule $I$, $p$ is a propositional variable and there is no right rule for $\nbang{}$ in $\aIMALLR$ since they only appear in negative polarity.
We shall write $\vdash_{\aIMALLR} S$ if the sequent $S$ is provable in $\aIMALLR$.\co{There is a problem with the rule for zero}

\begin{figure}[t]
\resizebox{\textwidth}{!}{
$
\begin{array}{c}
 \infer[\tensor_L]{\Gamma, A \tensor B \seq C}
{\Gamma, A, B \seq C} 
\quad
\infer[\tensor_R]{\nbang{}\Gamma,\Delta_1, \Delta_2 \seq A \tensor B}
{\nbang{}\Gamma,\Delta_1 \seq A & \nbang{}\Gamma,\Delta_2 \seq B}
\\\\
\infer[\lolli_L]{\nbang{}\Gamma,\Delta_1, \Delta_2, A \lolli B \seq C}
{\nbang{}\Gamma,\Delta_1 \seq A & \nbang{}\Gamma,\Delta_2, B \seq C}
\quad 
\infer[\lolli_R]{\Gamma \seq A \lolli B}{\Gamma, A \seq B}
\quad \infer[\nbang{}_L]{\Gamma,\nbang{a}A\seq C}{\Gamma,\nbang{a}A,A\seq C}
\\\\
 \infer[\with_{L_i}]{\Gamma, A_1 \with A_2 \seq B}
{\Gamma, A_i\seq B} 
\quad 
\infer[\with_R]{\Gamma \seq A \with B}
{\Gamma \seq A & \Gamma \seq B}
\quad
\infer[\oplus_L]{\Gamma, A \oplus B \seq C}
{\Gamma, A \seq C & \Gamma, B \seq C}
\quad 
\infer[\oplus_{R_i}]{\Gamma \seq A_1 \oplus A_2}{\Gamma \seq A_i}
\\\\
\infer[I]{\Gamma,p \seq p}{} 
 \qquad
\infer[\one_R]{\Gamma \seq \one}{}
\qquad 
\infer[\zero_L]{ \Gamma,\zero\nbang A}{}
\end{array}
$}\caption{Sequent system $\aIMALLR$}
\label{fig:ll}
%\vspace{-0.3cm}
\end{figure}

\subsection{Playing with subexponentials}
We shall  characterize $\aIMALLR$ proofs as winning strategies (w.s.) in a two-player game, the players denoted $\I$ and~$\II$. As usual, 
we will interpret bottom-up proof search in sequent systems as a game where,  at any given state, player \I first 
chooses a formula of a sequent and, in the next step,  either
\begin{itemize}
\item if the rule has
only one premise: \I moves to the premise sequent of the corresponding introduction rule; 
\item if the rule has two premises: 
\begin{enumerate}[i.]
\item player \II
chooses 
a premise sequent in which the game continues (if the rule has more than one premise).\co{This sentence in parenthesis is not needed}
\item  the game splits into independent subgames, all of
    which player \I has to win. \co{subgames, where \I has to win all of them if she want to win the game}
\end{enumerate}
\end{itemize}

The choice between $(i)$ and $(ii)$ depends on the nature of the rule:
branching
in {\em additive rules} is modeled as choices of \II, whereas in\co{Check this in... branching in??} branching {\em multiplicative rules},  \I   splits the context into two disjoint parts,
which then form the corresponding contexts of two  subgames to
be played in parallel. Consequently, a state of the game is represented
by a \textit{multiset of sequents}, each belonging to a separate subgame. 

Now, in order to capture the notion of {\em costs},  game states  will also involve a {\bf budget} (modeled as a real number) which will decrease whenever rule $\nbang{}_L$ is invoked\co{Invoked -> applied. Also, you can add ``This means that there is a `cost' $r$ in dereliction, i.e., when unpacking a formula `stored' in the modalitiy $\nbang{r}{}$}

Formally we have the following.
\begin{definition}[The game $\GAIMALLR$]\label{definition:GAIMALLR}
    $\GAIMALLR$ is a game of two players, \I and \II. Game states are tuples $(H,b)$, where $H$ is a finite multiset of extended sequents\co{Not defined yet} and $b\in\real$ is a ``budget''.
$\GAIMALLR$ proceeds in rounds, initiated by \I's selection of an extended sequent $S$ from the current game state. The successor state is determined according to rules that fit one of the two following schemes:

\[
\begin{array}{lllll}
{(1)} &(G\cup\{S\},b)&\quad\leadsto\quad&  \quad (G\cup\{S'\},b') & \\
{(2)} &(G\cup\{S\},b)&\quad\leadsto\quad&  \quad (G\cup\{S^1\}\cup\{S^2\},b)
\end{array}
\]

A round proceeds as follows: After \I has chosen an extended sequent $S\in H$ among the current game state, she chooses a rule  instance  $r$ of $\aIMALLR$ such that $S$ is the conclusion of that rule. Depending on  $r$, the round proceeds as follows:
\begin{enumerate}
\item If $r$ is a unary rule different from $\nbang{}_L$ with premise $S'$, then the game proceeds in the game state $(G\cup\{S'\},b)$.
\item {\bf Budget decrease:} If $r=\nbang{}_L$ with premise $S'$ and principal formula $\nbang{a}A$, then the game proceeds in the game state $(G\cup\{S'\},b-a)$.
\item {\bf Parallelism:} If $r$ is a binary rule with premises $S_1,S_2$ pertaining to a \emph{multiplicative} connective, then the game proceeds as $(G\cup\{S_1\}\cup\{S_2\},b)$.
\item {\bf \II-choice:} If $r$ is a binary rule with premises $S_1,S_2$ pertaining to an \emph{additive} connective, then \II chooses $S'\in\{S_1,S_2\}$ and the game proceeds in the game state $(G\cup\{S'\},b)$.
\end{enumerate}

\noindent A {\bf winning state} (for \I) is a game state $(H,b)$ such that all $S\in H$ are initial sequents of $\aIMALLR$ and $b\geq 0$.
\end{definition}

\begin{definition}[Plays and strategies]
A {\bf play} of $\GAIMALLR$ on a game state $(H,b)$ is a sequence $(H_1,b_1),(H_2,b_2),\ldots,(H_n,b_n)$ of game states, where $(H_1,b_1)=(H,b)$ and each $(H_{i+1},b_{i+1})$ arises by playing one round on $(H_i,b_i)$.
 A {\bf strategy} (for \I) on a game state $(H,b)$ is defined as a function telling \I how to move in any given state. 
 A strategy on $(H,b)$ is a {\bf winning strategy (w.s.)} if all plays following it eventually reach a winning state.
We write $\wins{}{\GAIMALLR}(H,b)$ if \I has a w.s. in the $\GAIMALLR$-game starting on $(H,b)$. 
\end{definition}
The intuitive reading of $\wins{}{\GAIMALLR}(H,b)$ is: \textit{The budget~$b$ suffices to win the game $H$.}

\begin{example}\label{ex:riddle}
Consider the following well-known riddle:
\begin{quote}
You have white and black socks in a drawer in a completely dark room. How many socks do you have to take out blindly to be sure of having a matching pair? 
\end{quote}
We can model the matching pair by the disjunction $(w\tensor w)\oplus(b\tensor b)$, and the act of drawing a random sock by the labelled formula $\nbang{1}(w\oplus b)$. The above question then becomes:
\begin{quote}
What is the least budget $n$  such that $\wins{}{\GAIMALLR}(\nbang{1}(w\oplus b)\seq (w\tensor w)\oplus(b\tensor b),n)$?
\end{quote}
The following play illustrate that $n=3$ suffices, where $F=(w\tensor w)\oplus(b\tensor b)$ and $G=\nbang{1}(w\oplus b)$: \\
\begin{enumerate}
\item $(\{G\seq F\},3)$ 
\item $(\{G, w\oplus b, w\oplus b, w\oplus b\seq F\},0)$ ($3\times$ budget decrease)
\item $(\{G, w, w\oplus b, w\oplus b\seq F\},0)$ ($\II$ chooses $w$)
\item $(\{G, w, b, w\oplus b\seq F\},0)$ ($\II$ chooses $b$)
\item $(\{G, w, b, b\seq F\},0)$ ($\II$ chooses $b$)
\item $(\{G, w, b, b\seq b\otimes b\},0)$ 
\item $(\{G, w, b\seq b\}\cup \{G, b\seq b\},0)$ (parallelism)
\end{enumerate}
Observe that the other possible choices for $\II$ are similar or simpler. 
\end{example}

\blue{EP. Stopped here.}

Taking these observations together, one can prove the following:

\begin{theorem}[Weak adequacy for $\GAIMALLR$]
\label{theorem:wadeq}
Let $S$ be a sequent. Then

$\exists b\left(\wins{}{\GAIMALLR}(\{S\},b)\right)\quad\iff\quad\vdash_{\aIMALLR} S$
\end{theorem}
The proof is similar to the one of Theorem \ref{theorem:adeq1}. We call this theorem \emph{weak} adequacy since information about the budget $b$ is lost in the proof theoretic representation. In other words, the game $\GAIMALLR$ is more expressive than the calculus $\aIMALLR$.
To overcome this discrepancy, we now introduce a  labelled extension of $\aIMALLR$ that we call $\laIMALLR$. A $\laIMALLR$-proof is build from labelled sequents $\Gamma\lra_b A$ where $\Gamma\lra A$ is an extended sequent and $b\in\real^+$. The complete system is given in Fig. \ref{fig:lll}. 
Our aim is to prove that
$\wins{}{\GAIMALLR}(\{\Gamma\lra A\},b)\quad\iff\quad\vdash_{\laIMALLR} \Gamma\lra_b A$.

\subsection{About cut-admissibility}\label{subsec:cut}
%!TEX root = CSL.tex

We begin by noting that establishing cut-admissibility in $\laIMALLR$ critically relies on the ability to define a computable function $f$ that relates the cost of the end-sequent to the labels of the premises in the cut rule.
%
Given that exponentials only occur negatively in $\laIMALLR$, no cut steps involve banged formulas. This enables us to demonstrate that $f(a,b) = a + b$ is the {\em minimal} such function.
\begin{theorem}[Negative-cut~\cite{DBLP:conf/tableaux/LangOPF19}]\label{thm:cutAdm}
For $f(a,b)=a+b$, the following cut rule is admissible in $\laIMALLR$:
$$
\infer[cut_\ell]{f(a,b): \nbang{}\Gamma,\Delta_1,\Delta_2\seq C}
	{a: \nbang{}\Gamma,\Delta_1\seq A &
	b:\nbang{}\Gamma,\Delta_2,A\seq C
	}
$$
Moreover, whenever $cut_\ell$ is admissible w.r.t. a given $f'$, then $a+b\leq f'(a,b)$.
\end{theorem}
It turns out that extending cost-conscious reasoning to modalities occurring {\em positively} in sequents is far from straightforward.
%
While an intuitive game-theoretic interpretation of promotion could be provided in the style of~\cite{DBLP:conf/tableaux/FermullerL17}, this {\em does not} align with a proof-theoretic notion of cut-admissibility. This is due to the inherent difficulty in defining a functional notion of the cut-label, as demonstrated below.

Let  
\laSELLR  be the system resulting from \laIMALLR~ by 
adding the following \emph{labelled promotion rule}
$$
\infer[\nbang{a}_R]{b:\Gamma\seq \nbang{a} A}{b:\Gamma^{\leqn{a}}\seq A}
$$
where $\Gamma^{\leqn{a}}$ denotes all formulas in $\Gamma$ which are of the form $\nbang{c} B$  and $a \geq c$. 

The question that arises is whether the cut-admissibility result can be extended to \laSELLR.
%
To address this, consider the following derivation:
$$
\infer[cut]{b_1+b_2+a:\,\Delta\seq C}
	{\infer[\nbang{a}_R]{b_1:\,\seq \nbang{a} A}{\deduce{b_1:\,\seq A}{}}
	&
	\infer[\nbang{a}_L]{b_2+a:\Delta,\nbang{a}A\seq C}
		{\deduce{b_2:\Delta,\nbang{a}A,A\seq C}{}}
	}
$$
This is usually reduced to
$$
\infer[cut]{2b_1+b_2:\Delta\seq C}
	{\deduce{b_1:\,\seq A}{} &
	\infer[cut]{b_1+b_2:\Delta,A\seq C}
	 	{\deduce{b_1:\,\seq \nbang{a} A}{} &
	 	\deduce{b_2:\Delta,\nbang{a}A,A\seq C}{}
	 	}
	}
$$
where the upper cut has a smaller rank, and the lower cut has a smaller degree than the original cut. However, this approach fails in the labelled setting because, whenever $a < b_1$, the label increases.

Although alternative reduction methods could be explored, the following result shows that it is impossible to define a labelled cut rule for \laSELLR\ where the label of the conclusion depends solely on the labels of the premises. We include the remarkable proof by Timo Lang\co{He is an author... better not to add this?}, as it is highly insightful.
\begin{theorem}[Impossible-cut~\cite{DBLP:conf/tableaux/LangOPF19}]\label{thm:impossible} There is no function $f:\real^+\times \real^+\rightarrow\real^+$ such that the rule
$$
\infer[cut]{f(a,b)\nbang{}\Gamma,\Delta_1,\Delta_2\seq C}
	{a: \nbang{}\Gamma,\Delta_1\seq A &
	b:\nbang{}\Gamma,\Delta_2,A\seq C
	}
$$
is admissible in $\laSELLR$.
\end{theorem}
\begin{proof}
Let $p,q$ be different propositional variables, and let $A^{\tensor n}$ denote the $n$-fold multiplicative conjunction of a formula $A$. The sequents
$$a:\nbang{1/k}p\seq\nbang{1/k}p^{\otimes (k\cdot a)}\qquad\text{and}\qquad b:\nbang{1/k}p^{\otimes (k\cdot a)}\seq p^{\otimes(k\cdot k\cdot a\cdot b)} 
$$
are provable in $\laSELLR$ for all natural numbers $a,b,k$. The smallest label~$f$ which makes their cut conclusion
$f: \nbang{1/k}p\seq p^{\otimes(k\cdot k\cdot a\cdot b)}
$ 
provable in~$\laSELLR$ is~$k\cdot a\cdot b$, which is not a function on the premise labels~$a,b$.
\end{proof}

\noindent
%While it is clear that if one can assign infinite costs to sequents then cut would be trivially admissible in $\laSELLR$, this still does not define a computable function relating the labels of the premises and the conclusion of the cut rule.
%In order to tackle this problem, we intend to mark cut-formulas with a certain {\em cost memory}, so to be able to keep track of accumulated costs.

The theorem above indicates that, to find an admissible labelled cut rule, we must either:
\begin{enumerate}
\item restrict the form of the cut formula;
\item allow the labelling function $f$ to incorporate more information from the premises than just their labels;
\item change the concept of cut. \co{this could be problematic} 
\end{enumerate}



We shall explore next
different fragments and (admissible) cut-like rules that can be proposed for such a calculus. 
\subsection{Infinite costs}
We start by observing that the inclusion of ``worse costs'' entails a trivial labelling 
that makes cut admissible. Let $\cRpi$ be the completion of $\cRp$ with $\infty$ and $\laSELLRi$ the corresponding labeled proof system with {\em decreasing} for $b\leq a$ being defined as follows:
\begin{itemize}
\item If $a,b\not=\infty$, $a - b$ is defined as usual;
\item If $a=\infty$, then $a - b=\infty$.
 \end{itemize}
In the following theorem,  the cut formula $F$ is an arbitrary formula (containing, possibly, positive and/or negative occurrences of 
the modality $\nbang{a}$). 

\begin{theorem}[Infinite-cut]
The following rule is admissible in $\laSELLRi$
$$
\infer[cut_\infty]{\infty: \nbang{}\Gamma,\Delta_1,\Delta_2 \seq C}{
 \deduce{a: \nbang{}\Gamma,\Delta_1 \seq F}{}&
  \deduce{b:\nbang{}\Gamma,\Delta_2, F \seq C}{}
 }
$$
\end{theorem}
The proof follows the same steps of the cut-elimination proof for $\SELL$~\cite{DBLP:conf/kgc/DanosJS93,DBLP:journals/jar/NigamM10}, 
using natural extensions of invertibility and permutability of rules to the labelled case.

Observe that this still does not define a computable function relating the labels of the premises and the conclusion of the cut rule.
\subsection{Linearity}
%It is worth noticing that the sole responsible for the impossibility result of Thm.~\ref{thm:impossible} is the explosive combination of the use of tensor/implication and contraction, that is, the multiplicative-sub-exponential fragment. Hence, limiting the occurrence of one or the other leads to more amenable results.

Now we show cases where the cut formula is restricted, starting with the case where the cut formula is $!$-free. 
\begin{theorem}[Linear-cut]
Let $A$ be a formula with no occurrences of 
$\nbang{a}$. Then, the following rule is admissible in $\laSELLR$
$$
\infer[cutL]{a + b:\nbang{}\Gamma,\Delta_1,\Delta_2\seq  C}
{\deduce{a: \nbang{}\Gamma,\Delta_1 \seq A}{}&
 \deduce{b: \nbang{}\Gamma,\Delta_2, A \seq C}{}&
}
$$
Moreover, if $a: \Gamma \seq C$ is provable using cutL, then there is a cut-free proof of 
$a': \Gamma \seq C$ with $a \geq a'$.
\end{theorem}
The proof uses a standard cut-reduction strategy for $\SELL$, observing in each case that the reduction of the label is possible. 

Still, forcing cut formulas to be linear seems to be a very severe restriction to impose. We will now consider another, and less limiting, syntactic restriction on the cut formula. 

\begin{definition} A formula of the form $\nbang{a} A $ is \emph{simply exp-labelled} if $a\neq 0$ and $A$ is bang-free.
\end{definition}

Since the formulas used in the proof of Theorem \ref{thm:impossible} can be simply exp-labelled, it is clear that we cannot expect to find an admissible cut rule for all simply exp-labelled cut formulas where the labelling depends solely on the labels of the premises. However, we can also incorporate the information from the label $a$ in the simply exp-labelled formula $\nbang{a} A$, as follows.

%First, one preliminary lemma.
%
%\begin{lemma}\label{lem:remove}
%If $\vdash_{\laSELLR} b:\Gamma,\nbang{a} A \seq C$ for some $b<a$, then $\vdash_{\laSELLR} b:\Gamma\seq C$.
%\end{lemma}
%\begin{proof}
%Let $\pr$ be a $\laSELLR$-proof of $b\:\Gamma,\nbang{a} A \seq C$ where $b<a$. Then every label in $P$ must be smaller than $a$, and so $\nbang{a}$ can never be principal in an application of $(\nbang{a}_L)$. Furthermore, since neither $\nbang{a} A $ nor $\nbang{a} A$ are atomic, they cannot appear in an initial sequent. It follows that we can simply remove the denoted occurrence of $\nbang{a} A $, as well as all its ancestors and applications of $(\nbang{a}_L)$ or $(weak)$ stemming from them, from $\pr$.
%\end{proof}

\begin{theorem}[Exp-labelled-cut~\cite{Timo-PhD}]\label{theorem:cut}
For any simply exp-labelled formula $\nbang{a} A $, the following cut rule is admissible in $\laSELLR$:
$$
\infer[cut_{el}]{f(b_1,b_2,a):\,!\Gamma,\Delta_1,\Delta_1\seq \Pi}{b_1:\,!\Gamma,\Delta_1\seq  \nbang{a} A  & b_2:\,!\Gamma,\Delta_2,\nbang{a} A \seq \Pi}
$$
where $f(b_1,b_2,a)=b_2+\lfloor b_2/a \rfloor\cdot b_1$.
\end{theorem}
The intuition behind this labelling is as follows: if the right subproof $R$ of the $cut_{el}$ ends with the label $b_2$, then the formula $\nbang{a} A$ can be unpacked at most $\lfloor b_2/a \rfloor$ times within a multiplicative subtree of $R$. Therefore, we can assume that the rule $\nbang{a}_L$ is applied only $\lfloor b_2/a \rfloor$ times on such a subtree.

%The proof of this theorem can be found in~\cite{}.
\subsection{Accumulated costs}
We will end the part of substructural modalities with a new approach towards cut-admissibility, given by keeping an exact track of the use of contraction in the cut-elimination process.  
The idea is that, if proving $A$ costs $b$, then any use of $A$  
must pay this ``extra cost''. In order to keep track of this extra cost, we introduce the following notation.

\begin{definition}
Let $\cE=\{a_b\mid a,b\in \cRpi\}$ be such that 
\begin{enumerate}
\item $a_b+_\cE c_d=a+b+c+d$.
\item $a_b \geq_\cE a_c$ (i.e., the ordering $\geq_\cE$ ignores the subindices).
\item $a_b >_\cE c_d$ iff $a>c$.
\end{enumerate}
For any formula $A\in\laSELLR$, we define $[A]_c$ as the formula that substitutes any 
modality $\nbang{a_b}{}$ with $\nbang{a_{b+c}}$.
\end{definition}
Hence $\laSELLRi$ can be slightly modified so that sequent labels belong to  $\cRpi$, while modal labels belong to $\cE$. Due to the ordering above, the promotion of $\nbang{a_0}$ 
has the same effect/constraints that the promotion of $\nbang{a_b}$. However, the dereliction of the latter requires a greater budget ($a+b$ instead of $a$). Moreover, the equivalence $\nbang{a_b}A \equiv \nbang{a_c}A$ can be proven, each direction requiring a different budget.
Finally, note that $\cE_0=\{a_0\mid a\in \cRpi\}\simeq \cRpi$, that is, each element $a\in\cRpi$ can be seen as the equivalence class of $a_0$ in $\cRpi\times \cRpi$ modulo $\cRpi$.
We will abuse the notation and continue representing the resulting system by $\laSELLRi$, also unchanging the representation of sequents. 


The following lemma has a straightforward proof.
\begin{lemma}
If $b:\Gamma, [A]_c \seq C$ then 
$b': \Gamma, A \seq C$
with $b \geq b'$. More generally,  if $b:\Gamma, [A]_c \seq C$ and $c \geq c'$ then
$b':\Gamma, [A]_{c'} \seq C$ with $b \geq b'$. 
\end{lemma}
The next definition restricts the appearance of unbounded modalities 
only under linear implication.
\begin{definition}
We say that $A$ is  $\limp$-linear if for all subformulas of the form $B \limp C$ in $A$, $B$ is bang-free.
%does not have occurrences of $\nbang{a}$. 
\end{definition}
The following result presents the admissibility of an extended form of the cut rule, where the budget information from the left premise is passed to the cut-formula in the right premise. Observe that the label
of the conclusion is now a function of the labels of the premises. Moreover, the cut-reduction is {\em label preserving}, meaning that the budget monotonically decreases in the cut-elimination process.
\begin{theorem}[$\limp$-linear-cut]
The following rule is admissible
$$
\infer[cut_{LL}\mbox{\quad $A$ is a $\limp$-linear formula}]{a+b: \nbang{}\Gamma,\Delta_1,\Delta_2 \seq C}{
 \deduce{a:\nbang{}\Gamma,\Delta_1 \seq A}{}&
 \deduce{b:\nbang{}\Gamma,\Delta_2, [A]_a \seq C}{}&
}
$$
Moreover, if $b: \Gamma \seq C$ is provable using $cut_{LL}$, then there is a cut-free proof of 
$b': \Gamma \seq C$ with $b\geq b'$.

\end{theorem}
\begin{proof}
We will illustrate some cases. 
\begin{itemize}
 \item Note that: $[\nbang{a_b}A]_c = \nbang{a_{b+c}}[A]_c$;  the promotion of $\nbang{a_b}A$, bottom-up, results in a context of 
 $\nbang{}$ formulas (that can be contracted at will);
 and   the dereliction of $\nbang{a_b}[A]_c$ decreases the budget in $a + b$. Hence, 
  $$
 \infer{a + b + 2c+d: \nbang{}\Gamma,\Delta_1,\Delta_2 \seq C}{
   \infer{c: \nbang{}\Gamma,\Delta_1 \seq \nbang{a_b}A}{c: (\nbang{}\Gamma)^\leqn{a_b}\seq A}&
   \infer{a+b+c+d: \nbang{}\Gamma, \Delta_2, \nbang{a_{b+c}}[A]_c \seq C}{d: \nbang{}\Gamma, \Delta_2, [A]_{c}, \nbang{a_{b+c}}[A]_c \seq C}
 }
 $$
reduces to
$$
   \infer{2c+d: \nbang{}\Gamma,\Delta_1,\Delta_2 \seq C}{
    \deduce{c: (\nbang{}\Gamma)^\leqn{a_b} \seq A}{} &
     \infer{c+d: \nbang{}\Gamma,\Delta_2, [A]_{c}\seq C}{
      \deduce{c:\nbang{}\Gamma \seq \nbang{a_b}A}{}&
      \deduce{d:\nbang{}\Gamma,\nbang{a_{b+c}}[A]_c, \Delta_2, [A]_{c}\seq C}{}
     }
    }
 $$
where the  ``extra cost'' $a_b$ disappears after the reduction. 
 \item Note that $[A\otimes B]_c = [A]_{c} \otimes [B]_c$. Here, let $c = c_1 + c_2$: 
 \[
 \infer{b+c: \nbang{}\Gamma,\Delta_1,\Delta_2 \seq C}{
  \infer{c:\nbang{}\Gamma,\Delta_1 \seq A \otimes B}{
   \deduce{c_1: \nbang{}\Gamma,\Delta_1' \seq A}{} &
   \deduce{c_2:\nbang{}\Gamma,\Delta_1'' \seq B}{} &
  } &
  \infer{b: \nbang{}\Gamma,\Delta_2, [A \otimes B]_{c} \seq C}{b: \nbang{}\Gamma,\Delta_2, [A]_c, [B]_c \seq C}
 }
 \]
 reduces to
\[
 \infer{b+c: \nbang{}\Gamma,\Delta_1,\Delta_2 \seq C}{
  \deduce{c_1: \nbang{}\Gamma,\Delta_1' \seq A}{} &
  \infer{b+c_2: \nbang{}\Gamma,\Delta_1 '',\Delta_2, [A]_{c_1} \seq C}{
   \deduce{c_2: \nbang{}\Gamma,\Delta_1'' \seq B}{} &
   \deduce{b: \nbang{}\Gamma,\Delta_2,[A]_{c_1}, [B]_{c_2} \seq C}{}
  }
 }
 \]
 It is worth noticing that in the first derivation, the cost $c=c_1 + c_2$ is ``charged'' to  $A\otimes B$ 
(in the formula $[A \otimes B]_c$)
while in the second one, in a  finer way, the cost $c_1$ is charged to $A$ and $c_2$ to $B$. 
 \item The case of implication explains the restriction we impose. Here $b = b_1 + b_2$:
 $$
\infer{c+b: \nbang{}\Gamma,\Delta_1,\Delta_2 \seq C }{
  \infer{c: \nbang{}\Gamma,\Delta_1 \seq A\limp B}{
   \deduce{c: \nbang{}\Gamma,\Delta_1,A  \seq B}{}&
   }&
   \infer{b: \nbang{}\Gamma,\Delta_2, [A \limp B]_c \seq C }{
     \deduce{b_1: \nbang{}\Gamma,\Delta_2' \seq [A]_{c}}{}&
     \deduce{b_2: \nbang{}\Gamma,\Delta_2'', [B]_c\seq C}{}
   }
}
 $$
 reduces to 
 \[
 \infer{c+b: \nbang{}\Gamma,\Delta_1,\Delta_2 \seq C}{
   \deduce{b_1: \nbang{}\Gamma,\Delta_2' \seq A}{}&
   \infer{c+b_2: \nbang{}\Gamma,\Delta_1,\Delta_2'', [A]_{b_1} \seq C}{
    \deduce{c: \nbang{}\Gamma,\Delta_1, [A]_{b_1} \seq B}{} &
    \deduce{b_2: \nbang{}\Gamma,\Delta_2'', [B]_{c} \seq C}{}
   }
 }
 \]
Note that the reduction above is correct since $A$ does not have 
occurrences of $\nbang{a}$ and then  $[A]_c = [A]_{b_1}=A$. 
\end{itemize}
\end{proof}
%This kind of analysis seems to be related with {\em flowgraphs} in \MELL~\cite{DBLP:journals/tcs/Strassburger03,DBLP:journals/tocl/StrassburgerG11}. 
%

Regardless of the chosen approach to enable cut-admissibility, the need for compositionality in dialogue games driven by a cut rule is a topic worth exploring~\cite{dutilh18}.


\subsection{Discussion -- part I}\label{subsec:conc1}
%!TEX root = CSL.tex

This research line offers at least four promising directions for future exploration.

First, the work initiated in~\cite{DBLP:conf/tableaux/LangOPF19} highlights that our games and systems provide more precise control over resources appearing negatively in sequents, unlocking new opportunities for analyzing the problem of comparing proofs. For instance, studying proof costs in labeled calculi could reveal deeper links between labels and computational bounds~\cite{DBLP:journals/jfp/AccattoliGK20}. Similarly, examining the interplay between resource budgets and the complexity of the cut-elimination process, particularly within the multiplicative-(sub)exponential fragment, presents considerable opportunities~\cite{DBLP:journals/tcs/Strassburger03,DBLP:journals/tocl/StrassburgerG11}.

Second, there is substantial value in investigating how the dialogue games we
have developed align with the framework of concurrent
games~\cite{DBLP:conf/lics/AbramskyM99,DBLP:conf/lics/FaggianM05,DBLP:journals/lmcs/CastellanCRW17}.
Understanding these connections could enrich our framework and provide new
perspectives on resource management in proof theory.

\blue{EP. Semirings?}\co{Check this and notice the ``four'' directions above}

Lastly, an essential direction involves addressing compositionality in dialogue games governed by the cut rule. Regardless of the specific approach taken to achieve cut-admissibility, ensuring compositionality remains a critical and promising challenge~\cite{dutilh18}.

\section{A game model for polarisation}
%!TEX root = CSL.tex

We now turn to the study of modalities in the classical setting, focusing on the positive-negative modal logic \PNL\ with nominals \cite{DBLP:journals/jolli/XiongA20,DBLP:journals/logcom/PedersenSA21}. This logic is based on Kripke frames with two disjoint and symmetric reachability relations. We will briefly outline the construction of an adequate semantic game for \PNL, its transformation into a provability game, and the derivation of a corresponding sequent system, concluding with a discussion.

In addition to the current roles of the players and the current formula $F$, it is necessary to keep track of the current world 
$w$ in the model. As a result, the game tree depends not only on the syntax of the formula but also on the relational structure of the model.

This behavior contrasts sharply with evaluation games for propositional logic, where semantic information is only required at the final stage to determine the winner. Furthermore, the technique employed in this paper, adapted from ~\cite{}, necessitates that the game tree for the evaluation game remains uniform across all models.

We address this challenge by allowing explicit references to worlds and the accessibility relation within the object language.

Let $\A=\{\ag,\b,\ldots\}$ be a non-empty set of agents,
$\At=\{p,q,\ldots\}$ be a countable set of propositional variables, and $N=\{i,j,\ldots\}$ be a countable set of \emph{nominals}. The language of \PNL~is generated by the following grammar
$$F  ::= p  \mid R^+(i,j)\mid R^-(i,j) \mid \neg F  \mid F_1  \wedge F_2  \mid F_1  \vee F_2  \mid \dplus F  \mid \dminus F \mid [A]F $$
where $p\in \At$, and $i,j\in N$. 
%The propositional connectives $\top$, $\bot$, $\to$, and the (dual)  modalities $\bplus$ and $\bminus$ can be obtained in the usual way. 

 Intuitively, the propositions $R^\pm(i,j)$ state that agent $i$ is a  \emph{friend}/\emph{enemy} with
 $j$. 
The formula $\dplus F $ (resp. $\dminus F $) states that $F $ holds for  a friend (resp.\ an enemy). The global
modality $[A]F $ states that $F $ holds for all the agents. 
We use $R^\pm$ to denote either $R^+$ or $R^-$, and 
 $\dplusminus$ to denote either $\dplus$ or $\dminus$. 

A model $\mathbb{M}$ is a tuple $\langle \A,\R^+,\R^-,\V,\g\rangle$ where $\A$
is a set (of agents), $\g:N\rightarrow \A$ is called \emph{denotation
function}, $\R^+,\R^-\subseteq \A\times \A$, and $\V:\At\rightarrow
\mathcal{P}(\A)$. A model is a \PNL-model if 
$\R^+$ is reflexive,  and 
$\R^+$ and $\R^-$ are both symmetric and 
non-overlapping, \ie,  for all $\ag,\b\in \A$, $(\ag,\b)\notin \R^+$ or $(\ag,\b)\notin \R^-$. 
\subsection{Discussion -- part II}\label{subsec:conc2}
%!TEX root = CSL.tex

This work opens up several promising directions for future exploration.

In terms of variations on the present work, it would be interesting to explore extensions of \PNL\ that relax symmetry assumptions, enabling the representation of scenarios where an agent $a$ can influence the opinion of agent $b$, but not vice versa. Another potential direction involves incorporating the concept of a ``budget,'' as introduced in the game discussed in the first part of this paper~\cite{DBLP:conf/tableaux/LangOPF19}, to model situations where proponents and opponents operate under a limited amount of political capital. In such scenarios, adding or modifying relationships could reduce this capital.

To formalize this idea, preferences for minimizing the expenditure of political capital could be expressed through a combination of \PNL\ with a suitable choice logic -- a framework where preferences are explicitly definable at the object level. Semantic games for choice logics have been explored in~\cite{Freiman2023TruthLogic}, and the extension of game-induced choice logic (\textbf{GCL}) to a provability game and proof system was proposed in~\cite{Freiman2023}. 
%Investigating these integrations offers a compelling avenue for further research.

Another particularly interesting avenue is extending the semantic-provability-proof system approach to other logics characterized by Kripke semantics. For instance, it would be worthwhile to investigate games for logics that involve model-change modalities~\cite{DBLP:journals/logcom/Velazquez-Quesada17,DBLP:journals/igpl/PerrotinV21} or dynamic modalities~\cite{DBLP:journals/synthese/BenthemGL08}. Initial progress in this direction was made in~\cite{LPAR2024:Reasoning_About_Group_Polarization}, where we showed how the global link-adding and local link-changing modalities from~\cite{DBLP:journals/logcom/PedersenSA21} (inspired by sabotage modal logic~\cite{DBLP:journals/igpl/ArecesFH15,DBLP:journals/logcom/AucherBG18,DBLP:journals/logcom/BenthemLSY23}) can be incorporated into our framework.

This extension is motivated by the study of social learning and opinion dynamics, which aim to understand how specific social factors influence the acceptance or rejection of opinions. Such models can provide insights into scenarios like consensus, polarization, and fragmentation. In these contexts, positive and negative relationships between agents are not fixed; they evolve over time. For example, enemies may reconcile, new friendships or agreements may form, or agents may develop disagreements. Exploring these dynamics within our framework offers a compelling direction for future research. 

Finally, we would like very much to investigate the availability of using this framework to propose games to constructive/intuitionistic modal logics. In particular, for the constructive logic $\CK$~\cite{DBLP:conf/eumas/AcclavioC23}
\bibliography{references}
\end{document}

\end{document}
