%!TEX root = CSL.tex

We now turn to the study of modalities in the classical setting, focusing on the positive-negative modal logic \PNL\ with nominals \cite{DBLP:journals/jolli/XiongA20,DBLP:journals/logcom/PedersenSA21}. This logic is based on Kripke frames with two disjoint and symmetric reachability relations. We will briefly outline the construction of an adequate semantic game for \PNL, its transformation into a provability game, and the derivation of a corresponding sequent system, concluding with a discussion.

In addition to the current roles of the players and the current formula $F$, it is necessary to keep track of the current world 
$w$ in the model. As a result, the game tree depends not only on the syntax of the formula but also on the relational structure of the model.

This behavior contrasts sharply with evaluation games for propositional logic, where semantic information is only required at the final stage to determine the winner. Furthermore, the technique employed in this paper, adapted from ~\cite{}, necessitates that the game tree for the evaluation game remains uniform across all models.

We address this challenge by allowing explicit references to worlds and the accessibility relation within the object language.

Let $\A=\{\ag,\b,\ldots\}$ be a non-empty set of agents,
$\At=\{p,q,\ldots\}$ be a countable set of propositional variables, and $N=\{i,j,\ldots\}$ be a countable set of \emph{nominals}. The language of \PNL~is generated by the following grammar
$$F  ::= p  \mid R^+(i,j)\mid R^-(i,j) \mid \neg F  \mid F_1  \wedge F_2  \mid F_1  \vee F_2  \mid \dplus F  \mid \dminus F \mid [A]F $$
where $p\in \At$, and $i,j\in N$. 
%The propositional connectives $\top$, $\bot$, $\to$, and the (dual)  modalities $\bplus$ and $\bminus$ can be obtained in the usual way. 

 Intuitively, the propositions $R^\pm(i,j)$ state that agent $i$ is a  \emph{friend}/\emph{enemy} with
 $j$. 
The formula $\dplus F $ (resp. $\dminus F $) states that $F $ holds for  a friend (resp.\ an enemy). The global
modality $[A]F $ states that $F $ holds for all the agents. 
We use $R^\pm$ to denote either $R^+$ or $R^-$, and 
 $\dplusminus$ to denote either $\dplus$ or $\dminus$. 

A model $\mathbb{M}$ is a tuple $\langle \A,\R^+,\R^-,\V,\g\rangle$ where $\A$
is a set (of agents), $\g:N\rightarrow \A$ is called \emph{denotation
function}, $\R^+,\R^-\subseteq \A\times \A$, and $\V:\At\rightarrow
\mathcal{P}(\A)$. A model is a \PNL-model if 
$\R^+$ is reflexive,  and 
$\R^+$ and $\R^-$ are both symmetric and 
non-overlapping, \ie,  for all $\ag,\b\in \A$, $(\ag,\b)\notin \R^+$ or $(\ag,\b)\notin \R^-$. 