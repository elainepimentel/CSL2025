%!TEX root = CSL.tex

This work opens up several promising directions for future exploration.

It would be interesting to explore extensions of \PNL\ that relax symmetry assumptions, enabling the representation of scenarios where an agent $a$ can influence the opinion of agent $b$, but not vice versa. Another potential direction involves incorporating the concept of a ``budget,'' as introduced in the game discussed in the first part of this paper, to model situations where proponents and opponents operate under a limited amount of \emph{political capital}. In such scenarios, adding or modifying relations (i.e., making new friends, making enemies to reconcile, etc) could reduce this capital.
Preferences on how to ``expend'' the political capital could be expressed through a combination of \PNL\ with a suitable choice logic -- a framework where preferences are explicitly definable at the object level. Semantic games for choice logics have been explored in~\cite{Freiman2023TruthLogic}, and the extension of game-induced choice logic (\textbf{GCL}) to a provability game and proof system was proposed in~\cite{Freiman2023}. 
Exploring these dynamics within our framework
offers a compelling direction for future research.

%Investigating these integrations offers a compelling avenue for further research.

Another particularly interesting avenue is extending the semantic-provability-proof system approach to other logics characterized by Kripke semantics. For instance, it would be worthwhile to investigate games for logics that involve model-change modalities~\cite{DBLP:journals/logcom/Velazquez-Quesada17,DBLP:journals/igpl/PerrotinV21} or dynamic modalities~\cite{DBLP:journals/synthese/BenthemGL08}. Initial progress in this direction was made in~\cite{LPAR2024:Reasoning_About_Group_Polarization}, where we showed how the global link-adding and local link-changing modalities from~\cite{DBLP:journals/logcom/PedersenSA21} (inspired by sabotage modal logic~\cite{DBLP:journals/igpl/ArecesFH15,DBLP:journals/logcom/AucherBG18,DBLP:journals/logcom/BenthemLSY23}) can be incorporated into our framework.

%This extension is motivated by the study of social learning and opinion dynamics, which aim to understand how specific social factors influence the acceptance or rejection of opinions. Such models can provide insights into scenarios like consensus, polarization, and fragmentation. In these contexts, positive and negative relationships between agents are not fixed; they evolve over time. For example, enemies may reconcile, new friendships or agreements may form, or agents may develop disagreements. Exploring these dynamics within our framework offers a compelling direction for future research. 

We are also  interested in exploring the application of this framework to develop games for constructive and intuitionistic modal logics~\cite{Fitch,plotkin:stirling:86,Sim94,DBLP:journals/sLogica/BiermanP00}. The constructive logic $\CK$ stands out as a promising candidate due to its intuitive semantics and straightforward sequent system. The main challenge lies in adapting the classical approach presented here to an intuitionistic setting. %The framework introduced in the first part of this paper could rove instrumental in addressing this challenge.

Building on ideas from~\cite{DBLP:conf/eumas/AcclavioC23}, we aim to establish a correspondence between winning innocent strategies in games played on Hyland-Ong arenas~\cite{DBLP:journals/iandc/HylandO00} and proofs in these constructive logics. This correspondence would deepen the connection between game semantics and constructive modal reasoning, opening new avenues for further study.

 
